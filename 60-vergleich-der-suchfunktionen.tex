\chapter{Vergleich der Suchfunktionen}
\label{chap:vergleich-der-suchfunktionen}

Zum Vergleich der beiden Suchfunktionen wird im Folgenden ein fiktionales Szenario dargestellt.
Das Szenario beschreibt den Ablauf des Onboardings eines neuen Mitarbeiters.
Dabei werden die expliziten Fragen beschrieben, welche sich der Mitarbeiter stellt, um sich einzuarbeiten.
Es wird angenommen, dass der Mitarbeiter sich mithilfe der Wissensdatenbank einarbeitet und die Suchfunktion der Wissensdatenbank verwendet.
Die Fragen, welche sich der Mitarbeiter des fiktionalen Szenarios stellt, werden anschließend bei der Durchführung einer Studie verwendet.
Die Studie verwendet die Fragen, um die Performance zwischen Suchfunktionen zu vergleichen.
Der genaue Aufbau der Studie ist im Folgenden näher beschrieben.
Nachdem der Aufbau der Studie erklärt wurde, werden die Daten der Studie ausgewertet und dargestellt.
Das Ergebnis wird diskutiert.
Zuletzt wird auf die Validität der Daten eingegangen und es wird der Versuchsaufbau diskutiert.

\section{Aufbau der Studie}
Im Rahmen der Bachelorarbeit wird eine Studie durchgeführt, welche die neue Suchfunktion mit der bestehenden Confluence-Suche vergleicht.
Da der Zeitrahmen begrenzt ist, in welchem die neue Suchfunktion entwickelt wird, wird nicht die tatsächliche Suchfunktion von Confluence entwickelt.
Es wurde bereits erwähnt, dass die Suche von Confluence auf Apache Lucene basiert, und dass dieses ein VSM auf Basis von BM25 verwendet.
Daher wird die neu entwickelte Suchfunktion in mehreren Konfigurationen verglichen.
So wird die neue Suchfunktion so konfiguriert, dass diese eine reine BM25 Suche durchführt.
Diese Konfiguration soll als Benchmark dienen und die tatsächliche Confluence-Suche repräsentieren.
Eine weitere Konfiguration verwendet eine Mischung aus einer BM25 Suche, und einer semantischen Suche.
Die beiden Suchalgorithmen werden zu gleichen Teilen verwendet.
Zuletzt verwendet eine andere Konfiguration lediglich die semantische Suche auf Grundlage von LSE.\\

Die Eigenschaft, welche untersucht werden soll ist, ob eine semantische Suche für den gegebenen Datensatz, und im Kontext einer Wissensdatenbank in der Softwareentwicklung, ebenfalls bessere Suchergebnisse liefert, als eine Suche auf Basis von BM25.
Durch die Verwendung einer einheitlichen Implementierung wird sichergestellt, dass lediglich die Effektstärken der Suchalgorithmen untersucht werden.
Es wird damit verhindert, dass die Confluence-Suche besser abschneidet, weil sie ausgereifter ist.
Es wird ebenfalls sichergestellt, dass die Datensätze der Suchfunktionen identisch sind.\\

Für die Studie wurde ein Datensatz generiert, welcher Dokumente beinhaltet, welche durch die Suchfunktion gefunden werden sollen.
Für jedes Dokument sind eine oder mehrere Sucheingaben definiert, mit dessen Eingabe das Dokument gefunden werden soll.
Darüber hinaus ist für jedes Dokument festgehalten, zu welchem Anwendungsfall sich dieses zuordnen lässt.\\

Die Suchfunktion gibt für jeden Algorithmus fünf Dokumente als Antwort auf eine Sucheingabe zurück.
Diese fünf Dokumente nach dessen Score sortiert.
Das bedeutet, dass das erste Dokument der Liste jenes ist, welches von dem Algorithmus als das passendste erachtet wird.
Die fünf Dokumente werden darauf untersucht, ob sich das gewünschte Dokument unter den Dokumente befindet.
Das Ergebnis ist ein Precision-Score für den Suchalgorithmus.
Um eine detailliertere Analyse zu ermöglichen wird nicht nur die Precision in Bezug auf die ersten fünf Dokumente gemessen.
Es wird die Precision für das erste Dokument, die ersten drei Dokumente und alle fünf Dokumente gemessen.
Anschließend werden die Precision-Scores der Algorithmen miteinander verglichen.\\

Die Studie wird vollkommen automatisch durchgeführt.
Dadurch können Ergebnisse der Studie nicht durch Teilnehmer verfälscht werden.
Es bedeutet auch, dass die Studie eine hohe Reliabilität hat und mit jeder Durchführung das gleiche Ergebnis liefert.
Sie ist Reproduzierbar.
Der Nachteil dieser Herangehensweise ist, dass die subjektive Wahrnehmung des Nutzers, in Bezug auf die Precision der Suchfunktion, nicht beachtet werden kann.
So ist es denkbar, dass eine Sucheingabe nicht das gewünschte Dokument beinhaltet, aber andere Dokumente, welche ein Nutzer als sinnvoll erachten würde.
Die Ergebnisse der Studie sind damit abhängig von der Vorauswahl der Dokumente, welche gefunden werden sollen, und der Sucheingaben, welche a priori als sinnvoll bestimmt wurden.
Es ist möglich, dass eine Suchfunktion für eine Sucheingabe durchaus ein sinnvolles Ergebnis liefert, aber nicht das Ergebnis, welches durch den Aufbau der Studie erwartet wird.

\section{Auswertung der Ergebnisse}
TODO: Ergänzen

\section{Diskussion des Studienaufbaus}

Wie bereits beschrieben gibt der Recall an, wie viele der relevanten Dokumente gefunden wurden.
Die Precision gibt lediglich an, wie viele der Dokumente, welche gefunden wurden, relevant sind.
Eine optimale Suchfunktion würde per Definition alle Dokumente finden, welche relevant sind, und keine unrelevanten Dokumente.
Umgekehrt ist eine schlechte Suchfunktion eine Suchfunktion, welche keine relevanten Dokumente findet, sondern nur unrelevante.
Dabei spielt es für den Nutzer aber keine Rolle, ob unrelevante Dokumente gefunden wurden.
Die Tatsache, dass die relevanten Dokumente nicht gefunden wurden, machen die Suchfunktion für den Nutzer nicht benutzbar. 
Die Studie untersucht die Precision der Suchalgorithmen.
Auf Grundlage der obigen Argumentation zeigt sich, dass der Recall-Score wichtiger ist als der Precision-Score.
Denn ist der Recall gering, dann ist die Suchfunktion nicht benutzbar.
Eine Suchfunktion mit einem hohen Recall, aber eine niedrigen Precision könnte dagegen viele relevanten Dokumente finden, aber auch viele irrelevante Dokumente.
Solange die relevanten Dokumente in der Liste weiter oben dargestellt werden, ist diese Zusammensetzung aus Precision und Recall gut.\\

Die Studie hat die Precision für das erste, die ersten drei und alle fünf Dokumente gemessen.
Die Studie hat allerdings keinen Recall gemessen, obwohl es sinnvoll ist den Recall als den wichtigeren Score einer Suchfunktion zu erachten.
Grund dafür ist die Tatsache, dass um den Recall messen zu können, alle relevanten Dokumente für eine Sucheingabe bekannt sein müssen.
Um für eine Studie a priori Sucheingaben zu bestimmen, und alle Dokumente, welche für diese Sucheingabe relevant sind, müsste der gesamte Datensatz bekannt sein.
Da der Datensatz mehrere hundert Dokumente beinhaltet ist dies nicht möglich.
Und auch wenn der gesamte Datensatz bekannt wäre, dann müsste trotzdem eine Entscheidung darüber getroffen werden, welche Dokumente relevant sind und welche nicht.
Das wäre wiederum eine subjektive Entscheidung, sodass die Validität des Ergebnisses anzweifelbar wäre.

TODO: Question Answering (Frage: was sind die dateiformate für den auftragseingang?/was für dateien gebe ich in den auftragseingang?, Antwort: excel - dateien oder csv - dateien)