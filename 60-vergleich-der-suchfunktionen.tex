\chapter{Vergleich der Suchfunktionen}
\label{chap:vergleich-der-suchfunktionen}

Zum Vergleich der beiden Suchfunktionen wird im Folgenden ein fiktionales Szenario dargestellt.
Das Szenario beschreibt den Ablauf des Onboardings eines neuen Mitarbeiters.
Dabei werden die expliziten Fragen beschrieben, welche sich der Mitarbeiter stellt, um sich einzuarbeiten.
Es wird angenommen, dass der Mitarbeiter sich mithilfe der Wissensdatenbank einarbeitet und die Suchfunktion verwendet der Wissensdatenbank verwendet.
Die Fragen, welche sich der Mitarbeiter des fiktionalen Szenarios stellt, werden anschließend bei der Durchführung einer Studie verwendet.
Die Studie verwendet die Fragen, um die Performance zwischen Suchfunktionen zu vergleichen.
Der genaue Aufbau der Studie ist im Folgenden näher beschrieben.
Nachdem der Aufbau der Studie erklärt wurde, werden die Daten der Studie ausgewertet und dargestellt.
Das Ergebnis wird diskuttiert.
Zuletzt wird auf die Validität der Daten eingegangen und es wird der Versuchsaufbau diskutiert.

\section{Fiktionales Szenario: Onboarding eines Mitarbeiters}

Für den Vergleich von Suchfunktionen müssen auf Grundlage der Anwendungsfälle realistische Fragestellungen entwickelt werden.
Zu diesem Zweck sei angenommen, dass ein neuer Softwareentwickler in einem bestehenden Softwareprojekt eingearbeitet wird.
Es wird also der Anwendungsfall des Onboardings betrachtet.\\

Zuerst wird dem Softwareentwickler aufgetragen, sich das \textit{Getting Started} im Confluence durchzulesen.
Der Softwareentwickler stellt sich also die Fragen, \textbf{wo sich das Getting Started befindet}.\\

Um sich mit der Software vertraut zu machen, wird dem neuen Softwareentwickler anschließend aufgetragen, die Anwendung bei sich lokal zu starten.
Nachdem er sie gestartet hat, soll er sich mit der Funktionalität der Software vertraut machen.
Um die Anwendung lokal zu starten, \textbf{sucht der Softwareentwickler nach einer Installationsanleitung}.\\

Nachdem die Software installiert ist, startet er die Anwendung.
Dazu muss er sich anmelden.
Er möchte herausfinden \textbf{mit welchen Anmeldedaten er sich anmelden kann}.
Nachdem er sich angemeldet hat, möchte er sich eine Übersicht über die Funktionen der Software verschaffen.
Dazu sucht er nach den \textbf{Use-Cases} der Anwendung.
Nachdem er die Use-Cases gefunden hat, probiert er mehrere davon aus.\\

Einer der Use-Cases ist die Versendung von Korrespondenzen am Ende eines Workflows.
Er möchte diesen Use-Case genauer nachvollziehen.
Deshalb möchte er herausfinden \textbf{wie man eine Korrespondenz verschickt} und \textbf{wie man den Empfänger einer Korrespondenz einstellen kann}.\\

Ein weiterer Use-Case ist die Erstellung eines Auftrags.
Der Softwareentwickler möchte herausfinden, \textbf{wie ein Auftrag erstellt werden kann}.\\

Der Softwareentwickler möchte sich weiterhin noch technischer mit der Software auseinandersetzen, und möchte sich daher das genaue Datenmodell eines Auftrags ansehen.
Er \textbf{sucht also nach dem Datenmodell des Auftrags}.\\

Nachdem sich der Softwareentwickler eine Weile mit der Software auseinandergesetzt hat, wird ihm mitgeteilt, dass Mitarbeiter des Projektes in Teams eingeteilt sind.
Dabei gibt es drei Entwicklungsteams: Team Blau, Team Rot, Team Gold.
Ihm wird erklärt, dass er von nun an Teil von Team Blau sein wird.
Daher beschließt der neue Softwareentwickler herauszufinden, welche anderen Mitarbeiter auch Teil seines Teams sind.
Er \textbf{möchte herausfinden, wer alles zu Team Blau gehört}.\\

Weiterhin möchte er genauer verstehen, wie die Entwicklung in dem Projekt abläuft.
Er beschließt herauszufinden, \textbf{für welche Teile die unterschiedlichen Teams und Mitarbeiter zuständig sind}.

TODO: 10 weitere Fragestellungen

\section{Aufbau der Studie}
Im Rahmen der Bachelorarbeit soll eine Studie durchgeführt werden, welche die neue Suchfunktion mit der bestehenden Confluence-Suche vergleicht.
Bei der Darstellung der Ergebnisse der Studie dürfen keine Geschäftsgeheimnisse preisgegeben werden.
Die Ergebnisse der Studie dürfen also nur Zahlen enthalten, und nicht tatsächlich gefundene Informationen in der Wissensdatenbank.\\

Die Teilnehmer nehmen freiwillig an der Studie teil und entstammen aus genau einem Projekt der adesso SE.
Der Name des Projektes soll nicht benannt werden.
Für die Studie werden lediglich Softwareentwickler ausgewählt.
Bei der Einteilung der Softwareentwickler soll darauf geachtet werden, dass die Zeit, welche die Teilnehmer im Projekt verbracht haben, je Gruppe ähnlich ist.
Die Teilnahme an der Studie ist anonym.\\

Die Teilnehmer der Studie werden in zwei Gruppen eingeteilt.
Jeder Teilnehmer bekommt die Aufgabe eine spezifische Information herauszufinden.
Dabei wird die Zeit gemessen, wie lange der Teilnehmer gebraucht hat, um die Information zu finden.
Es wird angenommen, dass immer eine korrekte Information gefunden wird.
Im Vorfeld wird geprüft, ob die Wissensdatenbank gründsätzlich eine Antwort auf die Fragestellung liefern kann. 
Zusätzlich wird die Anzahl der gemachten Sucheingaben dokumentiert.
Es werden insgesamt folgende Daten gesammelt und ausgewertet:
-	TeilnehmerId (unabhängige Variable - UV)
-	FragenId (UV)
-	Suchfunktion (UV)
-	Benötigte Zeit (abhängige Variable - AV)
-	Anzahl von Sucheingaben (AV)\\

Das Problem bei einer AB-Studie könnte sein, dass die Teilnehmer unterschiedliche Kenntnisse über die Wissensdatenbank haben, sodass einige Teilnehmer besser darin sind Informationen zu finden als andere.
Befinden sich in einer Gruppe eher Teilnehmer, welche bessere Kenntnisse über die Wissensdatenbank haben, und in der anderen Gruppe Teilnehmer, welche eher schlechtere Kenntnisse haben, dann werden die Ergebnisse der Studie verfälscht.
Mithilfe einer AB/BA-Studie könnte dieses Problem gelöst werden.
Dabei muss darauf geachtet werden, dass die Teilnehmer bei der Verwendung der zweiten Suchfunktion nicht die gleichen Informationen finden müssen, wie bei der ersten Suchfunktion.
Denn nachdem sie die Informationen mit der ersten Suchfunktion gefunden haben, wissen sie bereits, wo sich die Information befindet.
Es muss also zwei verschiedene Datensets geben.
Um Ungleichheiten zwischen den Datensets zu vermeiden, soll das Ursprungsdatenset randomisiert in zwei gleich große Subdatensets aufgeteilt werden.
Die erste Gruppe bekommt das erste Datenset für die Confluence-Suche und das zweite Datenset für die neue Suchfunktion.
Die zweite Gruppe bekommt das erste Datenset für die neue Suchfunktion und das zweite Datenset für die Confluence-Suche.\\

Weiterhin kann es bei den Teilnehmern zu Lerneffekten kommen, sodass die Verwendung der jeweils zweiten Suchfunktion besser funktioniert als die Verwendung der jeweils ersten Suchfunktion.
Der Lerneffekt der Teilnehmer soll gemessen werden und bei der Auswertung beachtet werden.

\section{Durchführung der Studie}
Für die Durchführung der Studie wurde eine Benutzeroberfläche entwickelt, welche die neue Suchfunktion verwendet.
Die Benutzeroberfläche beinhaltet ein Suchfeld, welches für die Sucheingaben verwendet wird.
Bei Eingabe in das Suchfeld wird die semantische Suche durchgeführt, wenn mindestens drei Zeichen eingegeben wurden.
Solange das Suchfeld weiterhin mindestens drei Zeichen beinhaltet, wird die Suche mit jeder Änderung der Eingabe erneut durchgeführt.
Damit werden auch die Suchergebnisse aktualisiert.\\

Die Suchergebnisse werden in einem Dropdown unterhalb der Sucheingabe dargestellt.
Die Darstellung soll jener von Confluence ähnlich aussehen.
Nachdem ein Suchergebnis aus dem Dropdown angeklickt wurde, wird die Seite unterhalb der Sucheingabe dargestellt.


\section{Auswertung der Ergebnisse}
TODO: Ergänzen


\section{Diskussion des Studienaufbaus}
TODO: Ergänzen

TODO: Durch Unterschiede zwischen den beiden Oberflächen könnte das Ergebnis verfälscht werden. 
