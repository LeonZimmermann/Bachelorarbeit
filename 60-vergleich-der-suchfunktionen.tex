\chapter{Vergleich der Suchfunktionen}
\label{chap:vergleich-der-suchfunktionen}

Zum Vergleich der beiden Suchfunktionen wird im Folgenden ein fiktionales Szenario dargestellt.
Das Szenario beschreibt den Ablauf des Onboardings eines neuen Mitarbeiters.
Dabei werden die expliziten Sucheingaben beschrieben, die der neue Mitarbeiter in der Wissensdatenbank macht, um sich einzuarbeiten.
Die Ergebnisse dieser Sucheingaben werden im nächstes Schritt dargestellt und ausgewertet.
Die zugrundeliegenden Daten sind im Anhang zu finden.
Zuletzt wird auf die Validität der Daten eingegangen.
Es wird der Versuchsaufbau diskutiert.

\section{Fiktionales Szenario: Onboarding eines Mitarbeiters}

Für den Vergleich von Suchfunktionen müssen auf Grundlage der Anwendungsfälle realistische Suchanfragen entwickelt werden.
Zu diesem Zweck sei angenommen, dass ein neuer Softwareentwickler in einem bestehenden Softwareprojekt eingearbeitet wird.
Es wird also der Anwendungsfall des Onboardings betrachtet.\\

Zuerst wird dem Softwareentwickler aufgetragen, sich das \textit{Getting Started} im Confluence durchzulesen.
Er macht also die Sucheingabe \textbf{Getting Started}, und erwartet ein Dokument mit ebendieser Überschrift.

TODO: Sucheingabe weiter beschreiben

Um sich mit der Software vertraut zu machen, wird dem neuen Softwareentwickler aufgetragen, die Anwendung bei sich lokal zu starten.
Nachdem er sie gestartet hat, soll er sich mit der Funktionalität der Software vertraut machen.
Um die Anwendung lokal zu starten, sucht der Softwareentwickler nach einer \textbf{Installationsanleitung}.
Er gibt also genau dies als Sucheingabe ein und erwartet als Ergebnis ein Dokument, welches beschreibt, wie die Software installiert wird.
TODO: Sucheingabe weiter beschreiben\\

Nachdem die Software installiert ist, startet er die Anwendung.
Dazu muss er sich anmelden.
Er sucht also nach \textbf{Testdaten}, welche die Anmeldedaten enthalten.
Nachdem er sich angemeldet hat, möchte er sich eine Übersicht über die Funktionen der Software verschaffen.
Dazu sucht er nach den \textbf{Use-Cases} der Anwendung.
Nachdem er die Use-Cases gefunden hat, probiert er mehrere davon aus.\\

Einer der Use-Cases ist die Versendung von Korrespondenzen am Ende eines Workflows.
Um diesen Use-Case genauer nachvollziehen zu können gibt er den Suchbegriff \textbf{Korrespondenz} ein.

Ein weiterer Use-Case ist die Erstellung eines Auftrags.
Die genauen die Details der Software dürfen nicht geschildert werden.
Daher kann nicht weiter darauf eingegangen werden, was ein Auftrag genau ist.
Er macht die Sucheingabe \textbf{Auftrag erstellen}.\\

Nachdem ein Auftrag erstellt wurde, lassen sich die Daten eines Auftrages anpassen.
Unter anderem lässt sich die Email-Adresse ändern, an welche die Korrespondenz am Ende des oben genannten Workflows geschickt wird.

Der Softwareentwickler möchte herausfinden, wie er diese ändern kann.
Dazu macht er die Sucheingabe \textbf{Empfänger von Korrespondenz ändern}.\\

Der Softwareentwickler möchte sich weiterhin noch technischer mit der Software auseinandersetzen, und möchte sich daher das genaue Datenmodell eines Auftrags ansehen.
Er macht also die Sucheingabe \textbf{Datenmodell Auftrag}.\\

Nachdem sich der Softwareentwickler eine Weile mit der Software auseinandergesetzt hat, wird ihm mitgeteilt, dass Mitarbeiter des Projektes in Teams eingeteilt sind.
Dabei gibt es drei Entwicklungsteams: Team Blau, Team Rot, Team Gold.
Ihm wird erklärt, dass er von nun an Teil von Team Blau sein wird.
Daher beschließt der neue Softwareentwickler herauszufinden, welche anderen Mitarbeiter auch Teil seines Teams sind.
Er macht also die Sucheingabe \textbf{Team Blau}.\\

Weiterhin möchte er genauer verstehen, wie die Entwicklung in dem Projekt abläuft.
Er beschließt sich herauszufinden, für welche Teile die unterschiedlichen Teams und Mitarbeiter zuständig sind.
Dazu macht der neue Softwareentwickler die Sucheingabe \textbf{Zuständigkeiten im Projekt}.

\section{Aufbau der Studie}
Im Rahmen der Bachelorarbeit soll eine Studie durchgeführt werden, welche die neue Suchfunktion mit der bestehenden Confluence-Suche vergleicht.
Bei der Darstellung der Ergebnisse der Studie dürfen keine Geschäftsgeheimnisse preisgegeben werden.
Die Ergebnisse der Studie dürfen also nur Zahlen enthalten, und nicht tatsächlich gefundene Informationen in der Wissensdatenbank.\\

Die Teilnehmer nehmen freiwillig an der Studie teil und entstammen aus genau einem Projekt der adesso SE.
Der Name des Projektes soll nicht benannt werden.
Für die Studie werden lediglich Softwareentwickler ausgewählt.
Bei der Einteilung der Softwareentwickler soll darauf geachtet werden, dass die Zeit, welche die Teilnehmer im Projekt verbracht haben, je Gruppe ähnlich ist.
Die Teilnahme an der Studie ist anonym.\\

Die Teilnehmer der Studie werden in zwei Gruppen eingeteilt.
Jeder Teilnehmer bekommt die Aufgabe eine spezifische Information herauszufinden.
Dabei wird die Zeit gemessen, wie lange der Teilnehmer gebraucht hat, um die Information zu finden.
Außerdem wird dokumentiert, ob die Information korrekt ist.
Zusätzlich wird die Anzahl der gemachten Sucheingaben dokumentiert.
Es werden insgesamt folgende Daten gesammelt und ausgewertet:
-	TeilnehmerID
-	AufgabenID
-	Suchfunktion
-	Benötigte Zeit
-	Anzahl von Sucheingaben
-	Korrektheit der Antwort\\

Das Problem bei einer AB-Studie könnte sein, dass die Teilnehmer unterschiedliche Kenntnisse über die Wissensdatenbank haben, sodass einige Teilnehmer besser darin sind Informationen zu finden als andere.
Befinden sich in einer Gruppe eher Teilnehmer, welche bessere Kenntnisse über die Wissensdatenbank haben, und in der anderen Gruppe Teilnehmer, welche eher schlechtere Kenntnisse haben, dann werden die Ergebnisse der Studie verfälscht.
Mithilfe einer AB/BA-Studie könnte dieses Problem gelöst werden.
Dabei muss darauf geachtet werden, dass die Teilnehmer bei der Verwendung der zweiten Suchfunktion nicht die gleichen Informationen finden müssen, wie bei der ersten Suchfunktion.
Denn nachdem sie die Informationen mit der ersten Suchfunktion gefunden haben, wissen sie bereits, wo sich die Information befindet.
Es muss also zwei verschiedene Datensets geben.
Um Ungleichheiten zwischen den Datensets zu vermeiden, soll das Ursprungsdatenset randomisiert in zwei gleich große Subdatensets aufgeteilt werden.
Die erste Gruppe bekommt das erste Datenset für die Confluence-Suche und das zweite Datenset für die neue Suchfunktion.
Die zweite Gruppe bekommt das erste Datenset für die neue Suchfunktion und das zweite Datenset für die Confluence-Suche.\\

Weiterhin kann es bei den Teilnehmern zu Lerneffekten kommen, sodass die Verwendung der jeweils zweiten Suchfunktion besser funktioniert als die Verwendung der jeweils ersten Suchfunktion.
Der Lerneffekt der Teilnehmer soll gemessen werden und bei der Auswertung beachtet werden.


\section{Auswertung der Ergebnisse}
TODO: Ergänzen


\section{Diskussion des Studienaufbaus}
TODO: Ergänzen
