\chapter{Vergleich der Suchfunktionen}
\label{chap:vergleich-der-suchfunktionen}
Für den Vergleich von Suchfunktionen müssen auf Grundlage der Anwendungsfälle realistische Suchanfragen entwickelt werden.
Zu diesem Zweck sei angenommen, dass ein neuer Softwareentwickler in einem bestehenden Softwareprojekt eingearbeitet wird.
Es wird also der Anwendungsfall des Onboardings betrachtet.\\

Zuerst wird dem Softwareentwickler aufgetragen, sich das \textit{Getting Started} im Confluence durchzulesen.
Er macht also die Sucheingabe \textbf{Getting Started}, und erwartet ein Dokument mit ebendieser Überschrift. 
TODO: Sucheingabe weiter beschreiben

Um sich mit der Software vertraut zu machen, wird dem neuen Softwareentwickler aufgetragen, einmal die Anwendung bei sich lokal zu starten.
Nachdem er sie gestartet hat, soll er sich mit der Funktionalität der Software vertraut machen.
Um die Anwendung lokal zum Laufen zu bringen, sucht der Softwareentwickler nach einer \textbf{Installationsanleitung}.
Er gibt also genau dies als Suchbegriff ein und erwartet als Ergebnis ein Dokument, welches beschreibt, wie die Software installiert wird.
TODO: Sucheingabe weiter beschreiben\\

Nachdem die Software installiert ist, startet er die Anwendung.
Dazu muss er sich anmelden.
Er sucht also nach \textbf{Testdaten}, welche die Anmeldedaten enthalten.
Nachdem er sich angemeldet hat, möchte er sich eine Übersicht über die Funktionen der Software verschaffen.
Dazu sucht er nach den \textbf{Use-Cases} der Anwendung.
Nachdem er die Use-Cases gefunden hat, probiert er mehrere davon aus.\\

Einer der Use-Cases ist die Versendung von Korrespondenzen am Ende eines Workflows.
Um diesen Use-Case genauer nachvollziehen zu können gibt er den Suchbegriff \textbf{Korrespondenz} ein.



Im Anhang ist eine Auflistung der beispielhaften Sucheingabe zu finden.

\section{Diskussion des Studienaufbaus}
TODO: Ergänzen

\section{Durchführung der Studie}
TODO: Ergänzen

\section{Diskussion der Ergebnisse}
TODO: Ergänzen
