%- Was ist das generelle Thema der Arbeit?
%Beispiel: Kontext ist der Schnittpunkt Variabilitätsmodellierung / SPLE / SA.
%
%- Was ist das ganz konkrete Problem, das in der Arbeit behandelt wird? Was ist die Forschungslücke?
%Beispiel: DFDs haben keine Variabilitätserweiterung. Bisherige Lösungen behandeln das Thema nur abstrakt und können eine Konsistenzprüfung im Sinne des SPLE-Gedankens nicht konkret umsetzen.
%
%- Was ist die generelle Vorgehensweise (Methodik, z.B. Art der Studie: Experiment, Literatursuche, …)?
%Beispiel: SLR zum Aufspüren von Problemen und potentiellen Lösungen mit dem Ziel, diese zu einem eigenen Ansatz zu kombinieren.
%
%- Was ist das zentrale Resultat? Was folgt aus dem Resultat (Wert der Arbeit, z.B. für künftige Studien)?
%Beispiel: Ansatz zur SA im SPLE mit Toolunterstützung. Macht den Weg frei für weiterführende Ansätze in dem Rahmen (evtl. ein knackiges Beispiel nennen, was man tun kann).
%
%Dabei sollte die Wortwahl – im Gegensatz zu den von mir gegebenen Stichpunkten – natürlich durchgehend knackig und äußerst präzise sein. Zu jedem Punkt sollten jeweils ein bis zwei Sätze formuliert werden.
%Insgesamt sollte ein Abstract ca. 100-250 Wörter lang sein (grob eine eine halbe Seite bei dem üblichen Format, niemals mehr als eine Seite).
%
%Du solltest dabei daran denken, dass ein Abstract potentiellen Lesern zur Grobeinordnung Deiner Arbeit dient, z.B., um festzustellen, ob die Arbeit für ihre Forschung relevant sein könnte. Außerdem ist ein Abstract das Aushängeschild der Arbeit; er sollte bei am Problem interessierten Lesern (auch bei Gutachtern!) großes Interesse wecken.
%%
%
%
%
\section*{Abstract}
TODO: Zusammenfassung auf Englisch

%
%
%
\section*{Zusammenfassung}
Softwareentwickler müssen sich bei ihrer Arbeit Informationen zusammensuchen, welche sie für die weitere Arbeit benötigen.
So muss ein Softwareentwickler bei der Implementierung eines Features die Intention des Features kennen.
Solche Informationen können in Wissensdatenbanken hinterlegt sein.
Um dort die gewünschten Informationen zu finden können Suchfunktionen verwendet werden.
Insbesondere, wenn dem Softwareentwickler im Vorfeld nicht klar ist, wo er die gewünschten Informationen in der Wissensdatenbank finden kann.
Aber nicht immer liefert diese Suchfunktion die gewünschten Informationen.\\

In dieser Arbeit wird eine semantische Suchfunktion entwickelt.
Diese soll eine Verbesserung zu bestehenden Suchfunktionen bieten, wie beispielsweise die Suchfunktion von Confluence.
Um festzustellen, ob die neue Suchfunktion \textit{besser} ist als die bestehende, werden Methoden und Kriterien zur Evaluierung von Suchfunktionen erörtert.
Anhand dieser Kriterien werden die neu implementierte Suchfunktion und die bestehende Suchfunktion von Confluence in einer Studie verglichen.
In der Studie werden beispielhafte Sucheingaben definiert, sowie die erwarteten Ergebnisse.
Die Definition der Sucheingaben erfolgt auf Basis von Anwendungsfällen.
Die Anwendungsfälle beschreiben Situationen, in denen es realistisch ist, dass ein Softwareentwickler die Suchfunktion einer Wissensdatenbank verwendet.
Anhand von Argumentationen werden entsprechend realistische Sucheingaben erstellt.\\

TODO: Ergebnis der Studie darstellen