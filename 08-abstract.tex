%- Was ist das generelle Thema der Arbeit?
%Beispiel: Kontext ist der Schnittpunkt Variabilitätsmodellierung / SPLE / SA.
%
%- Was ist das ganz konkrete Problem, das in der Arbeit behandelt wird? Was ist die Forschungslücke?
%Beispiel: DFDs haben keine Variabilitätserweiterung. Bisherige Lösungen behandeln das Thema nur abstrakt und können eine Konsistenzprüfung im Sinne des SPLE-Gedankens nicht konkret umsetzen.
%
%- Was ist die generelle Vorgehensweise (Methodik, z.B. Art der Studie: Experiment, Literatursuche, …)?
%Beispiel: SLR zum Aufspüren von Problemen und potentiellen Lösungen mit dem Ziel, diese zu einem eigenen Ansatz zu kombinieren.
%
%- Was ist das zentrale Resultat? Was folgt aus dem Resultat (Wert der Arbeit, z.B. für künftige Studien)?
%Beispiel: Ansatz zur SA im SPLE mit Toolunterstützung. Macht den Weg frei für weiterführende Ansätze in dem Rahmen (evtl. ein knackiges Beispiel nennen, was man tun kann).
%
%Dabei sollte die Wortwahl – im Gegensatz zu den von mir gegebenen Stichpunkten – natürlich durchgehend knackig und äußerst präzise sein. Zu jedem Punkt sollten jeweils ein bis zwei Sätze formuliert werden.
%Insgesamt sollte ein Abstract ca. 100-250 Wörter lang sein (grob eine eine halbe Seite bei dem üblichen Format, niemals mehr als eine Seite).
%
%Du solltest dabei daran denken, dass ein Abstract potentiellen Lesern zur Grobeinordnung Deiner Arbeit dient, z.B., um festzustellen, ob die Arbeit für ihre Forschung relevant sein könnte. Außerdem ist ein Abstract das Aushängeschild der Arbeit; er sollte bei am Problem interessierten Lesern (auch bei Gutachtern!) großes Interesse wecken.
%%
%
%
%
\section*{Abstract}
TODO

%
%
%
\section*{Zusammenfassung}
TODO