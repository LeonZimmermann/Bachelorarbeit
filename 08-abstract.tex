%- Was ist das generelle Thema der Arbeit?
%Beispiel: Kontext ist der Schnittpunkt Variabilitätsmodellierung / SPLE / SA.
%
%- Was ist das ganz konkrete Problem, das in der Arbeit behandelt wird? Was ist die Forschungslücke?
%Beispiel: DFDs haben keine Variabilitätserweiterung. Bisherige Lösungen behandeln das Thema nur abstrakt und können eine Konsistenzprüfung im Sinne des SPLE-Gedankens nicht konkret umsetzen.
%
%- Was ist die generelle Vorgehensweise (Methodik, z.B. Art der Studie: Experiment, Literatursuche, …)?
%Beispiel: SLR zum Aufspüren von Problemen und potentiellen Lösungen mit dem Ziel, diese zu einem eigenen Ansatz zu kombinieren.
%
%- Was ist das zentrale Resultat? Was folgt aus dem Resultat (Wert der Arbeit, z.B. für künftige Studien)?
%Beispiel: Ansatz zur SA im SPLE mit Toolunterstützung. Macht den Weg frei für weiterführende Ansätze in dem Rahmen (evtl. ein knackiges Beispiel nennen, was man tun kann).
%
%Dabei sollte die Wortwahl – im Gegensatz zu den von mir gegebenen Stichpunkten – natürlich durchgehend knackig und äußerst präzise sein. Zu jedem Punkt sollten jeweils ein bis zwei Sätze formuliert werden.
%Insgesamt sollte ein Abstract ca. 100-250 Wörter lang sein (grob eine eine halbe Seite bei dem üblichen Format, niemals mehr als eine Seite).
%
%Du solltest dabei daran denken, dass ein Abstract potentiellen Lesern zur Grobeinordnung Deiner Arbeit dient, z.B., um festzustellen, ob die Arbeit für ihre Forschung relevant sein könnte. Außerdem ist ein Abstract das Aushängeschild der Arbeit; er sollte bei am Problem interessierten Lesern (auch bei Gutachtern!) großes Interesse wecken.
%%
%
%
%
\section*{Abstract}
Software developers often need to gather Information to continue with their work.
An example of this might be the implementation of a feature, where the software developer needs to understand the intent of the feature to be implemented.
Such information could be found in a knowledge base.
To find information in a knowledge base, search engines can be used.
This is especially helpful, when the software developer does not know beforehand, where the needed information is to be found in the knowledge base.
However, search engines don't always find the information that the software developer needs.\\

As part of this Bachelor's Thesis, a sematic search engine is developed.
The development of a semantic search engine aims to be an improvement on existing search engines, which use TF-IDF or BM25 for the scoring of documents.
An example of such search engines is the search engine of the knowledge base Confluence.
To determine whether or not the newly developed search engine resembles an \textit{improvement} over the existing search engine of Confluence, this thesis discusses methods and metrics for the evaluation of search engines.
The search engine of Confluence and the newly developed search engine are then compared on the basis of the previously discussed methods and metrics.
For that, exemplary queries are formulated, as well as the expected documents to be returned by the search engines.
The queries are based on Use Cases that have been previously formulated.
These Use Cases describe situations, in which it is realistic for a software developer to be using the search engine of a knowledge base.\\

The study showed, that for the generated dataset the sematic search has a lower precision value compared to a BM25 search.
It is argued, that the results can be explained by the fact, that the dataset that was used, belongs to a specific domain and contains many technical terms from that specific domain.
Now, a sentence transformer, which is used to index the documents for the semantic search, is trained on a general purpose dataset containing general purpose terms only.
That said, the sentence transformer could not have learned the semantic of the technical terms from that specific domain.
Finally the Bachelor's Thesis mentions approaches to adapt sentence transformers to a specific domain, increasing the precision of the semantic search.

\section*{Zusammenfassung}
Softwareentwickler müssen sich bei ihrer Arbeit Informationen zusammensuchen, welche sie für die weitere Arbeit benötigen.
So muss ein Softwareentwickler bei der Implementierung eines Features die Intention des Features kennen.
Solche Informationen können in Wissensdatenbanken hinterlegt sein.
Um dort die gewünschten Informationen zu finden können Suchfunktionen verwendet werden.
Insbesondere, wenn dem Softwareentwickler im Vorfeld nicht klar ist, wo er die gewünschten Informationen in der Wissensdatenbank finden kann.
Aber nicht immer liefert diese Suchfunktion die gewünschten Informationen.\\

In dieser Arbeit wird eine semantische Suchfunktion entwickelt.
Diese soll eine Verbesserung zu bestehenden Suchfunktionen bieten, welche TF-IDF oder BM25 für das Scoring von Ergebnissen verwenden, wie beispielsweise die Suchfunktion von Confluence.
Um festzustellen, ob die neue Suchfunktion \textit{besser} ist als die bestehende, werden Methoden und Kriterien zur Evaluierung von Suchfunktionen erörtert.
Anhand dieser Kriterien werden die neu implementierte Suchfunktion und die bestehende Suchfunktion von Confluence in einer Studie verglichen.
In der Studie werden beispielhafte Sucheingaben definiert, sowie die erwarteten Ergebnisse.
Die Definition der Sucheingaben erfolgt auf Basis von Anwendungsfällen.
Die Anwendungsfälle beschreiben Situationen, in denen es realistisch ist, dass ein Softwareentwickler die Suchfunktion einer Wissensdatenbank verwendet.\\

Die Studie hat gezeigt, dass für den generierten Datensatz eine semantische Suche eine geringere Precision besitzt, als eine BM25 Suche.
Das Ergebnis wird damit begründet, dass es sich bei dem Datensatz um eine spezifische Domäne mit vielen Fachbegriffen handelt.
Ein Sentence Transformer, welcher dazu verwendet wird die Daten für die semantische Suche zu indizieren, wurde nur auf allgemeinen Datensätzen mit allgemeinen Begriffen trainiert.
Dadurch hat der Sentence Transformer noch nicht die Semantik der Domäne gelernt.
Zuletzt nennt die Arbeit Ansätze, um Sentence Transformer an eine spezifische Domäne anzupassen, und die Precision der semantischen Suche damit zu erhöhen.