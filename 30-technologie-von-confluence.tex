\chapter{Technologien von Confluence}
\label{chap:technologie-von-confluence}
Dieses Kapitel hat den Zweck die Funktionsweise der Suchfunktion von Confluence zu erläutern.
Dies ist notwendig, um später die Ergebnisse des Vergleichs der Suchfunktion von Confluence mit der neuen Suchfunktion zu verstehen.
Um die Suchfunktion von Confluence zu verstehen wird zuerst erklärt, welche Suchmethoden mit den Technologien von Apache Lucene ermöglicht werden und in Confluence verwendet werden können.
Dann wird die zugehörige Technologie erklärt, indem zuerst ein Überblick über die wichtigsten Komponenten einer Suchfunktion gegeben wird.
Als nächstes werden die spezifischen Technologien erläutert, welche Apache Lucene verwendet.
Das bedeutet, dass zuerst erklärt wird, was ein Volltext-Index ist.
Anschließend werden die Scoring Algorithmen TF-IDF und BM25 erörtert.

\section{Suchmethoden von Confluence}
\label{chap:suchmethoden-von-confluence}
Dieses Kapitel stellt die Arten von Sucheingaben vor, welche durch Apache Lucene ermöglicht werden.
Es ist wichtig diese zu kennen, um das Qualitätskriterium später bei der Implementierung erfüllen zu können.
Es gibt die gängigen Suchalgorithmen Keyword Search, Phrase Search, Boolean Search, Field Search.\\

Eine Keyword Search durchsucht Dokumente nach der Sucheingabe des Nutzers.
Die Eingabe wird dabei nicht als Ganzes betrachtet, sondern jedes Wort einzeln.
Für jedes Keyword werden Dokumente als Ergebnis angezeigt, wenn dieses in dem Dokument vorhanden ist.
Wenn ein Dokument mehrere der Keywords beinhaltet, wird dessen Relevanz höher eingeschätzt als für Dokumente, welche weniger Keywords enthalten.
Dokumente mit höherer Relevanz werden weiter oben in der Ergebnisliste angezeigt.\\

Eine Phrase Search ist die Suche nach Textausschnitten in Dokumenten.
Hier werden nicht mehrere Keywords einzeln betrachtet, sondern die gesamte Eingabe in das Suchfeld als eine Einheit.
Es reicht also nicht mehr aus, dass ein Dokument eines der Wörter enthält.
Es muss die gesamte Sucheingabe als ein String enthalten sein.\\

Die Boolean Search bietet die Möglichkeit einen boolschen Ausdruck als Sucheingabe zu machen.
Ein Beispiel dafür ist die Sucheingabe \textit{Dokumentation AND Angular}.
Die Sucheingabe bedeutet, dass die Suchfunktion nur Dokumente als Ergebnis darstellen soll, welche beide Keywords Dokumentation und Angular enthalten.
Die Boolean Search kann auch eine Phrase, wie bei der Phrase Search, beinhalten: \textit{"Dokumentation von Software" AND Angular}.
In diesem Beispiel werden nur Dokumente als Ergebnis nur angezeigt, wenn sie den gesamten String \textit{Dokumentation von Software} enthalten, sowie das Keyword \textit{Angular}.
Bei einer Boolean Search können die boolschen Operatoren \textit{AND}, \textit{OR}, \textit{NOT} beliebig kombiniert werden.\\

Eine Field Search sucht Dokumente anhand von Attributen.
Der Nutzer kann diese Attribute auswählen.
Wenn der Nutzer beispielsweise ein Dokument sucht, welches am 01.01.2005 erstellt wurde, dann kann die Eingabe der Suche so aussehen: \textit{erstelldatum: 01.01.2005}.
Es können beliebig viele Attribute verwendet werden, um die Suche einzugrenzen.
Neben der Verwendung der Attribute für die Suche selbst, können die Attribute komplementär zu einer anderen Art von Suche verwendet werden.
So kann eine Suchfunktion Buttons bereitstellen, über welche Filter festgelegt werden.
Nun kann eine Keyword Search durchgeführt werden, aber die gefundenen Dokumente werden mithilfe der Filter weiter eingeschränkt.
Neben diesen gängigen Suchalgorithmen gibt es weitere Suchalgorithmen, wie die strukturierte Suche und die semantische Suche.

\section{Architektur einer Suchfunktion}
Dieses Kapitel gibt einen Überblick über die wichtigstes Komponenten einer Suchfunktion.
Eine einfache Implementierung einer Suchfunktion kann aus drei Komponenten bestehen.
Aus dem Crawling, dem Index und dem Scoring-Algorithmus.
Das Crawling ist zuständig für das Finden von Dokumenten (\cite{Castillo_2005}).
Der Index speichert Informationen der Dokumente, und die Suche ist zuständig für das Verstehen der Nutzeranfrage und die Abfrage der relevantesten Informationen aus dem Index, sowie dessen Verarbeitung und Darstellung.
Der Begriff Dokument wird hier verwendet, um die Dateien zu beschreiben, welche durch einen Crawler gesucht und durch den Index verarbeitet werden.
Unter Dokumenten können auch eine Website verstanden werden, welche durch einen Webcrawler durchsucht werden.
Für die Implementierung der Suche wird ein Scoring-Algorithmus benötigt.
Der Scoring-Algorithmus bestimmt, welche Dokumente am besten auf eine Sucheingabe passen.\\

Für die Implementierung einer Suchfunktion wird zunächst ein Datensatz von Dokumenten benötigt, welche über die Suchfunktion gefunden werden können.
Dieser wird mithilfe eines Crawlers aufgebaut.
Ein Crawler ist ein Algorithmus, welcher Techniken aus dem Natural Language Processing nutzt, um Informationen aus einem Dokument zu extrahieren (\cite{Khder_2021}).
Implementierungen können reguläre Ausdrücke verwenden, um die Informationen zu extrahieren, oder auch fortgeschrittenere Verfahren, wie Abstract Syntrax Trees.
Im Falle von Websites kann der Crawler Hyperlinks zu weiteren Websites extrahieren.
Damit kann der Algorithmus sukzessive den Datensatz von Dokumenten befüllen.
Die neuen Dokumente werden durch den Index verarbeitet und wiederum auf neue Links analysiert.
Dieses Verfahren kann beliebig lange und beliebig rekursiv durchlaufen werden, um den Index zu erweitern.
Idealerweise, ohne Seiten dabei mehrfach zu durchlaufen.
Neben dem Crawling können Indizes befüllt werden, indem eine Liste von Dokumenten übergeben werden, welche dem Index hinzugefügt werden sollen.\\

Die Indizierung von Dokumenten kann sowohl mit einer Volltext-Indizierung oder auch einer Vektor-Indizierung umgesetzt werden.
Ein Volltextindex bestimmt den Score anhand von Ausschnitten aus dem Volltext.
Zum einen gibt es Boolean Models im Document Retrieval, welche boolsche Algebra nutzen, um relevante Dokumente zu bestimmen.
Das Ergebnis ist eine Boolean Search, wie in \myRef{Kapitel}{chap:suchmethoden-von-confluence} beschrieben.
Zum anderen gibt es Probablistische Modelle im Document Retrieval.
Ein Vektorindex berechnet Vektoren anhand des Ursprungstextes.
Es gibt zwei Modelle des Document Retrievals, welche Vektorindizes benutzen: Sparse Vectors und Deep Vectors.
Die Berechnung von Sparse Vectors kann mithilfe von tf-idf erfolgen, wie in \myRef{Kapitel}{chap:sparse-vector-models} beschrieben.
Die Berechnung von Deep Vectors kann wiederum mithilfe von Sentence Transformers erfolgen, wie in \myRef{Kapitel}{chap:sentence-transformer} beschrieben.

\section{Volltext-Indizierung}
\label{chap:volltext-indizierung}
Nachdem ein Überblick über die Komponenten einer Suchfunktion gegeben wurde, werden nun die Technologien erläutert, welche Apache Lucene bereitstellt.
Apache Lucene verwendet einen Volltext-Index.
Eine Möglichkeit zur Umsetzung einer Volltext-Indizierung ist der invertierte Index.
Zur Generierung des invertierten Indexes müssen die zu indizierten Dokumente zunächst in einer Datenbank abgelegt werden.
Ein invertierter Index wird generiert, indem alle Wörter extrahiert werden, welche in den Dokumenten vorkommen.
Nun werden diese Wörter in eine Liste geschrieben, und jedem Wort wird zugeordnet, in welchem Dokument sich dieses Wort wiederfinden lässt.
Diese Liste wird invertierter Index genannt, weil nicht die Wörter den Dokumenten zugeordnet sind, sondern die Dokumente den Wörtern.
Es wird ebenfalls gespeichert, an welchen Stelle des Dokuments das Wort vorkommt, und auch in wie vielen Dokumenten ein Wort vorkommt.\\

Bei der Indizierung der Wörter besteht nun die Problematik, dass gleiche Wörter in unterschiedlichen Formen existieren können.
So stammen \textit{Heizung} und \textit{heizen} beide von dem gleichen Wortstamm \textit{heiz} ab.
Um bei der Indizierung Speicherplatz zu sparen, können Wörter auf diesen Wortstamm reduziert werden, damit sie als ein einziges Wort betrachtet werden können.
Die Bildung des Wortstamms wird als Stemming bezeichnet.
Beim Stemming kann es jedoch zu Overstemming und Understemming kommen.
Overstemming bedeutet, dass zwei Wörter, die eigentlich nichts miteinander zu tun haben, also nicht semantisch gleich sind, den gleichen Wortstamm besitzen und als ein Wort betrachtet werden.
Ein Beispiel hierfür sind die Wörter \textit{Wand} und \textit{wandere}, wie in \textit{ich wandere}.
Beide besitzen den Wortstamm \textit{wand} und werden entsprechend als ein Wort betrachtet.
Understemming bedeutet, dass zwei Wörter, die eigentlich etwas miteinander zu tun haben, also semantisch gleich sind, nicht den gleichen Wortstamm besitzen und dadurch als zwei verschiedene Wörter betrachtet werden.
Ein Beispiel hierfür sind die Wörter \textit{absorbieren} und \textit{Absorption}, welche die Wortstämme \textit{absorb} und \textit{absorp} besitzen.
Es gibt Techniken zur Vermeidung solcher Probleme, wie der Einsatz vollständiger morphologischer Analysekomponenten.
Hierauf soll aber nicht weiter eingegangen werden.\\

\section{Sparse Vector Modelle}
\label{chap:sparse-vector-models}
Da nun die Dokumente indiziert sind, kann ein Scoring-Algorithmus bestimmen, welche Dokumente die passendsten für eine gegebene Sucheingabe sind.
Apache Lucene verwendet für das Scoring von Dokumenten die TF-IDF Metrik.
TF-IDF untersucht, wie häufig Terme in einem Dokument vorkommen und wie charakteristisch diese Terme für dieses Dokument sind (\cite{Manning_Raghavan_Schuetze_2019}).
Kommt ein gesuchter Term häufig in einem Dokument vor, dann erhöht dies den Score für dieses Dokument.
Kommt dieser Term selten in anderen Dokumente vor, dann erhöht dies den Score ebenfalls.
Umgekehrt ist der Score für ein Dokument niedriger, je seltener der gesuchte Term in dem Dokument vorkommt oder je häufiger der Term in allen anderen Dokumenten vorkommt.
Der tf-Teil steht für \textit{term frequency} und wird berechnet, indem für das jeweilige Dokument bestimmt wird, wie häufig das Wort in dem Dokument vorkommt.
Damit die Metrik nicht abhängig von der Länge des Dokumentes ist, wird dieser Wert durch die insgesamte Anzahl der Vorkommnisse des Wortes dividiert.
Damit ist diese Metrik relativ zur insgesamten Anzahl der Vorkommnisse des Wortes, und nicht absolut.
Der idf-Teil steht für \textit{inverse document frequency}.
Er wird berechnet indem die insgesamte Anzahl der Dokumente durch die Anzahl der Dokumente dividiert wird, welche das Wort enthalten.
Das Ergebnis des dividierens wird an die Logarithmus-Funktion übergeben, sodass am Ende der idf-Wert berechnet wurde.
Die beiden Werte werden miteinander multipliziert.
Das Ergebnis ist die TF-IDF-Metrik:

\[\text{tf-idf}_{t,d}=\text{tf}_{t,d} \times \text{idf}_t\]

\[\text{tf}_{t,D}=\frac {\#_{t,D}}{\max_{t'\in D}(\#_{t',D})}\]

\[\text{idf}_t=\log {\frac {N}{\text{df}_t}}\]

Neben TF-IDF-Metrik wurde die BM25-Metrik entwickelt, welche Apache Lucene ebenfalls bereitstellt.
BM25 ist eine probablistische Scoring Metrik.
Sie verwendet ebenfalls die inverse Dokumentenfrequenz zur Ermittlung des Scores von Dokumenten.
So wie bei der TF-IDF-Metrik stellt tf hier die Termfrequenz dar.
Die Variablen \(k_1\), \(k_3\) und b stellen Hyperparameter dar, welche entsprechend der Datenbank und der Sucheingaben angepasst werden können.
Die Variable dl ist die Länge des Dokumentes und die Variable avdl die durchschnittliche Länge der Dokumente.
Q ist die Sucheingabe, welche die Terme T enthält.
% TODO: Unklarheiten bei qtf
Die Variable qtf wird definiert als die Termfrequenz innerhalb einer Domäne, aus welcher sich die Sucheingabe ableitet (\cite{Beaulieu_Track_Sparck_Karen_Willett_Peter}):

\[\sum_{T \in Q}w^{(1)} \frac{(k_1 + 1)\text{tf}}{K + \text{tf}} \frac{(k_3 + 1)\text{qtf}}{k_3 + \text{qtf}}\]

\[K=k_1((1 - b) + b.\text{dl} / \text{avdl})\]

% TODO: Weiter schreiben