\chapter{Evaluationsmethoden und -Kriterien}
Um zu verifizieren, dass die Implementierung ihre gewünschte Wirkung erzielt, müssen zuerst Methoden und Kriterien zur Messung herangezogen werden.
Karl Popper gilt als Begründer des kritischen Rationalismus.
Es war der Anfang von wissenschaftlich durchgeführten Experimenten.
Bei einem Experiment wird zunächst eine Hypothese aufgestellt.
Diese wird anschließend durch das Experiment untersucht.
Ein solches Experiment soll später für die genannten Anwendungsfälle erstellt werden.
Dazu sollen für die Anwendungsfälle Sucheingaben erstellt werden.
Für jede Sucheingabe wird definiert, welche Dokumente als Ergebnis erwartet werden.
Die Hypothese: Mit der implementierten Suchfunktion werden die erwarteten Dokumente \textit{besser} gefunden als mit der bisherigen Suchfunktion.
Diese Hypothese muss nun quantifizierbar und messbar gemacht werden.
Die subjektive Wahrnehmung reicht nicht für eine wissenschaftliche Arbeit aus.\\

Die Formulierung der Hypothese ist für sich genommen zu unspezifisch.
Es muss geklärt werden, wann eine Information \textit{besser} zu finden ist.
Im Folgenden werden Precision, Recall und F-Maß, sowie Qualitätskriterien gemäß ISO/IEC 9126 herangezogen, um Suchfunktionen anhand dieser Eigenschaften vergleichbar zu machen.
Anhand dessen wird hergeleitet, was in diesem Kontext \textit{besser} bedeutet.
Im Anschluss werden diese Eigenschaften in einem Versuchsaufbau verwendet. 

\section{Precision, Recall und F-Maß}
Zur Evaluation der Suchfunktionen werden später die statistischen Messwerte Precision und Recall verwendet.
Die Messwerte beschreiben, inwieweit eine Hypothese zutrifft.
Wenn also die Hypothese ist, dass ein bestimmtes Dokument gefunden wird, dann ist der Precision-Wert das Verhältnis zwischen allen gefundenen Dokumenten und den gefundenen Dokumenten, die tatsächlich relevant sind.
Der Wert lässt sich im Kontext der Suche nach Dokumenten wie folgt definieren\cite{Sirotkin_2012}:\\

\(Precision=P=\frac{gefundene relevante Dokumente}{gesamte Anzahl gefundener Dokumente} \)\\

Der Recall-Wert gibt wiederum an, wie viele von den tatsächlich relevanten Dokumenten auch gefunden wurden.
Er lässt sich in diesem Kontext wie folgt definieren\cite{Sirotkin_2012}:\\

\(Recall=R=\frac{gefundene relevante Dokumente}{gesamte Anzahl relevanter Dokumente}\)\\

Es ist schwierig beide Werte zu optimieren, da der Precision-Wert versucht die Anzahl der gefundenen Dokumente einzugrenzen und der Recall-Wert versucht die Anzahl der gefundenen Dokumente zu erweitern.
Das F1-Maß fasst beide Werte zu einem neuen Wert zusammen\cite{Sirotkin_2012}:\\

\(F_1=2\frac{PR}{P+R}\)\\

Neben der Verwendung des F1-Maß ist es wichtig sich Gedanken darüber zu machen, welcher der beiden Messwerte wichtiger ist.
In diesem Fall ist es sinnvoll eher die Precision zu optimieren.
Denn, wenn ein Dokument gesucht wird, aber überhaupt nicht gefunden werden kann, dann erfüllt die Suchfunktion nicht ihren Zweck.
Wenn die Suchfunktion irrelevante Dokumente darstellt, kann sie trotzdem ihren Zweck erfüllen, solange die relevantesten Dokumente zuerst in der Liste der Ergebnisse dargestellt wird.
Dieser Faktor gilt auch bei der Implementierung zu berücksichtigen.

\section{ISO/IEC 9126}
Die ISO/IEC 9126 legen Qualitätskriterien für die Entwicklung von Software fest.
Es gibt sechs verschiedene Qualitätskriterien, welche sich wiederum in verschiedene Facetten einteilen lassen.
Doch nicht alle der sechs Qualitätskriterien sind in dieser Arbeit von Relevanz.
So werden Die Qualitätskriterien Wartbarkeit, Effizienz, Übertragbarkeit und Zuverlässigkeit nicht beachtet.
Die Wartbarkeit bezieht sich auf die Fähigkeit von Software sich zu verändern.
Diese Fähigkeit hat allerdings nichts mit der Qualität einer Suchfunktion zu tun, wie sie hier gemessen werden soll.
Die Effizienz ist im Kontext von Suchfunktionen als gegeben anzunehmen.
Der Author unterstellt hier, dass jede Suchfunktion, welche untersucht werden soll, auch effizient ist.
Denn im Jahr 2023 ist anzunehmen, dass eine Suchfunktion auch schnell die Suchergebnisse darstellen kann.
Die Übertragbarkeit beschreibt die Fähigkeit einer Software, auf verschiedenen Umgebungen lauffähig zu sein.
Die Zuverlässigkeit beschreibt, ob die Software auch bei unterschiedlichen Rahmenbedingungen in der gewünschten Form funktioniert.
Die beiden Kriterien liegen ebenfalls außerhalb des Scopes der Arbeit, weil auch sie sich nicht auf die Qualität der Suchfunktion als solches beziehen.\\

Die zu betrachtenden Qualitätskriterien sind Benutzbarkeit und Funktionalität.
Die beiden Qualitätskriterien werden in den nächsten Kapiteln im Detail erläutert.
Das schließt die verschiedenen Facetten der Qualitätskriterien mit ein.

\subsection{Benutzbarkeit}
Die Benutzbarkeit von Software teilt sich in die Facetten Attraktivität, Konformität, Erlernbarkeit, Verständlichkeit und Bedienbarkeit ein.
Die Attraktivität soll hier nicht weiter betrachtet werden, weil die reine Optik einer Suchfunktion im Kontext dieser Arbeit keine Relevanz besitzt.
Die Facetten Erlernbarkeit und Verständlichkeit ergeben sich durch die Konformität der Suche.
Damit eine Suchfunktion konform ist, muss sie alle gängigen Arten von Suchen unterstützen.
Also Keyword Search, Phrase Search, Boolean Search, Phrase Search.
Die unterschiedlichen Arten von Suchen werden in dem Kapitel "Theoretischer Hintergrund" näher erläutert.\\

Die übrige Facette ist die Bedienbarkeit.
Eine Autovervollständigung von Wörtern kann es dem Softwareentwickler vereinfachen, passende Sucheingaben zu machen.
Sie verhindert zum einen, dass er Tippfehler macht.
Zum anderen hilft sie dem Softwareentwickler auch bei der Wortwahl.
Sie sorgt also für eine bessere Bedienbarkeit.\\

\subsection{Funktionalität}
Die Funktionalität von Software teilt sich in die Facetten Angemessenheit, Sicherheit, Interoperabilität, Konformität, Ordnungsmäßigkeit und Richtigkeit auf.

TODO: Weiter ausführen

\section{Versuchsaufbau}
Für den Versuchsaufbau werden Dokumente benötigt, welche durchsucht werden können.
Dazu sollen Daten aus einem echten Projekt verwendet werden.
Da die Informationen zu dem Projekt nicht veröffentlicht werden dürfen, müssen die Sucheingaben und Ergebnisse so eingeschränkt werden, dass keine projektspezifischen Informationen preisgegeben werden.
Die Daten sind bereits in einem Confluence Space vorhanden, dessen Suche als Benchmark für die neu implementierte Suchfunktion verwendet wird.
Die Implementierung der neuen Suchfunktion und die Befüllung dessen Datenbank mit den gleichen Daten, wie die Confluence Suche, wird in dem Kapitel \textit{Implementierung} näher erklärt.\\

Wie zuvor erwähnt, müssen nun Sucheingaben erstellt werden, welche in beiden Suchen eingegeben werden können.
Zum Vergleich der beiden Suchfunktionen werden für jede Sucheingabe die obersten fünf Ergebnisse betrachtet.
Für jedes Dokument in diesen Ergebnissen, welches laut Experimentaufbau als Ergebnis erwartet wird, erhält die Suchfunktionen einen Hit.
Die Hits werden für beide Suchfunktionen zusammengezählt und verglichen.
Anschließend wird eine einfaktorielle Varianzanalyse mithilfe von SPSS durchgeführt.
Die Analyse wird zeigen, ob die neu implementierte Suchfunktion signifikant mehr Hits generiert als die bestehende Confluence Suche.

\section{Einfaktorielle Varianzanalyse}
Die Varianzanalyse überprüft, ob ein signifikant unterschiedlicher Wert zwischen den beiden Suchfunktionen besteht.
Wenn die neu implementierte Suchfunktion signifikant mehr Dokumente findet als die Confluence-Suche, dann bestätigt dies die Hypothese.
Die Hypothese ist dabei, dass die neu implementierte Suchfunktion eine Verbesserung zur bestehenden Confluence-Suche darstellt.\\

Zur Durchführung der einfaktoriellen Varianzanalyse muss die abhängige Variable und die unabhängige Variable definiert werden.
Die unabhängige Variable ist die verwendete Suchfunktion.
Diese muss numerisch repräsentiert werden.
So steht die Zahl 0 für die Confluence-Suche, und die Zahl 1 für die neu implementierte Suchfunktion.
Die abhängige Variable ist die Anzahl der gefundenen erwarteten Ergebnisse.
Sie ist die abhängige Variable, weil die Anzahl von der verwendeten Suchfunktion abhängt.
