\chapter{Evaluationsmethoden und -Kriterien}
Um zu verifizieren, dass die Implementierung ihre gewünschte Wirkung erzielt, müssen zuerst Methoden und Kriterien zur Messung herangezogen werden.
Karl Popper gilt als Begründer des kritischen Rationalismus.
Dieser war der Anfang von wissenschaftlich durchgeführten Experimenten.
Zunächst wird eine Hypothese aufgestellt.
Diese wird anschließend durch ein Experiment untersucht.
Ein solches Experiment kann ebenfalls für die genannten Anwendungsfälle erstellt werden.
Dazu soll die folgende Hypothese formuliert werden: Mit der vorgestellten Suchfunktion werden die, vom Softwareentwickler gesuchten Informationen, besser gefunden, als mit der bisherigen Suchfunktion.
Diese Hypothese muss nun quantifizierbar und messbar gemacht werden.
Die subjektive Wahrnehmung reicht nicht für eine wissenschaftliche Arbeit aus.\\

Nun ist die Formulierung der Hypothese für sich genommen zu unspezifisch.
Es muss geklärt werden, wann eine Information \textit{besser} zu finden ist.
Im Folgenden werden Precision, Recall und F-Maß, sowie Qualitätskriterien gemäß ISO/IEC 9126 herangezogen, um Suchfunktionen anhand dieser Eigenschaften vergleichbar zu machen.
Im Anschluss werden diese Eigenschaften in einem Versuchsaufbau verwendet. 

\section{Precision, Recall und F-Maß}
Zur Evaluation werden die Werte Precision und Recall gemessen.
Der Precision-Wert ist das Verhältnis zwischen allen gefundenen Dokumenten und den gefundenen Dokumenten, die tatsächlich relevant sind.
Der Recall-Wert gibt an, wie viele von den tatsächlich relevanten Dokumenten auch gefunden wurden.
Es ist schwierig beide Werte zu optimieren, da der Precision-Wert versucht die Anzahl der gefundenen Dokumente einzugrenzen und der Recall-Wert versucht die Anzahl der gefundenen Dokumente zu erweitern.
Das F-Maß fasst beide Werte zu einem neuen Wert zusammen.

\section{ISO/IEC 9126}
Die ISO/IEC 9126 legen Qualitätskriterien für die Entwicklung von Software fest.
Es gibt sechs verschiedene Qualitätskriterien, welche sich wiederum in verschiedene Facetten einteilen lassen.
Doch nicht alle der sechs Qualitätskriterien sind in dieser Arbeit von Relevanz.
So werden Die Qualitätskriterien Wartbarkeit, Effizienz, Übertragbarkeit und Zuverlässigkeit nicht beachtet.
Die Wartbarkeit bezieht sich auf die Fähigkeit von Software sich zu verändern.
Diese Fähigkeit hat allerdings nichts mit der Qualität einer Suchfunktion zu tun, wie sie hier gemessen werden soll.
Die Effizienz ist im Kontext von Suchfunktionen als gegeben anzunehmen.
Der Author unterstellt hier, dass jede Suchfunktion, welche untersucht werden soll, auch effizient ist.
Denn im Jahr 2023 ist anzunehmen, dass eine Suchfunktion auch schnell die Suchergebnisse darstellen kann.
Die Übertragbarkeit beschreibt die Fähigkeit einer Software, auf verschiedenen Umgebungen lauffähig zu sein.
Die Zuverlässigkeit beschreibt, ob die Software auch bei unterschiedlichen Rahmenbedingungen in der gewünschten Form funktioniert.
Die beiden Kriterien liegen ebenfalls außerhalb des Scopes der Arbeit, weil auch sie sich nicht auf die Qualität der Suchfunktion als solches beziehen.\\

Die zu betrachtenden Qualitätskriterien sind Benutzbarkeit und Funktionalität.
Die beiden Qualitätskriterien werden in den nächsten Kapiteln im Detail erläutert.
Das schließt die verschiedenen Facetten der Qualitätskriterien mit ein.

\subsection{Benutzbarkeit}
Die Benutzbarkeit von Software teilt sich in die Facetten Attraktivität, Konformität, Erlernbarkeit, Verständlichkeit und Bedienbarkeit ein.
Die Attraktivität soll hier nicht weiter betrachtet werden, weil die reine Optik einer Suchfunktion im Kontext dieser Arbeit keine Relevanz besitzt.
Die Facetten Erlernbarkeit und Verständlichkeit ergeben sich durch die Konformität der Suche.
Damit eine Suchfunktion konform ist, muss sie alle gängigen Arten von Suchen unterstützen.
Also Keyword Search, Phrase Search, Boolean Search, Phrase Search.
Die unterschiedlichen Arten von Suchen werden in dem Kapitel "Theoretischer Hintergrund" näher erläutert.\\

Die übrige Facette ist die Bedienbarkeit.
Eine Autovervollständigung von Wörtern kann es dem Softwareentwickler vereinfachen, passende Sucheingaben zu machen.
Sie verhindert zum einen, dass er Tippfehler macht.
Zum anderen hilft sie dem Softwareentwickler auch bei der Wortwahl.
Sie sorgt also für eine bessere Bedienbarkeit.\\

\subsection{Funktionalität}
Die Funktionalität von Software teilt sich in die Facetten Angemessenheit, Sicherheit, Interoperabilität, Konformität, Ordnungsmäßigkeit und Richtigkeit auf.

TODO Weiter ausführen

\section{Versuchsaufbau}
Für einen Versuchsaufbau werden Dokumente benötigt, welche durchsucht werden können.
Außerdem müssen die Suchanfragen festgelegt werden, welche durchgeführt werden sollen.
Für jede Suchanfrage muss das gewünschte Ergebnis angegeben werden.
Ein Versuchsaufbau vergleicht Treatments miteinander.
Zwischen den Treatments werden bestimmte Eigenschaften untersucht.
Hier wird als Eigenschaft untersucht, an welcher Stelle sich das gewünschte Ergebnis befindet.
Eine Information ist besser zu finden, wenn die Dokumente, welche die Information enthalten, weiter oben in den Ergebnissen angezeigt werden.
Eine Information ist ebenfalls besser zu finden, wenn der Softwareentwickler weniger Versuche bei der Formulierung der Suchanfrage benötigt, um die Information zu finden.
Außerdem ist eine Information besser zu finden, wenn ein größerer Anteil der gefundenen Ergebnisse von Relevanz sind.
Also wenn die Precision hoch ist.
Eine Information ist ebenfalls besser zu fidnen, wenn ein größerer Anteil aller relevanten Dokumente auch tatsächlich in den Ergebnissen aufgelistet sind.
Also wenn der Recall hoch ist.
Die Treatments sind in diesem Kontext die verschiedenen Suchfunktionen.