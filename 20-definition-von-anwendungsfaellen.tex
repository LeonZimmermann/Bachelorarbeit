\chapter{Definition von Anwendungsfällen}
\label{chap:anwendungsfaelle}
In diesem Kapitel werden Anwendungsfälle ausgewählt, welche eine Grundlage für die spätere Evaluation der Suchfunktionen bilden.
Die Anwendungsfälle beschreiben die Situationen, in denen Softwareentwickler die Suchfunktionen von Wissensdatenbanken verwenden könnten.
Zur Identifikation von Anwendungsfällen wurde zunächst Literatur herangezogen.
Die Literatur ist bereits unter den verwandten Arbeiten aufgeführt.
In den verwandten Arbeiten wurden Arbeiten genannt, welche ähnliche Probleme lösen sollen.
Diese fokussieren sich vorallem auf die \textit{Feature Location}, \textit{Bug Localization} und die Traceability zwischen Code und anderen Artefakten.
Aus diesen Themen ließ sich der Anwendungsfall identifizieren, welcher in \myRef{Kapitel}{chap:implementierung-nach-spezifikation} erörtert wird.
Außerdem konnten noch weitere Anwendungsfälle ermittelt werden.
Dazu wurde die Wissensdatenbank eines Softwareprojektes untersucht.
So ist das Onboarding im Projekt ein Anwendungsfäll für die Verwendung der Suchfunktion einer Wissensdatenbank.
Ein weiterer Anwendungsfall ist das Auffinden von Informationen über das Projektmanagement.\\

Die folgenden Kapitel beschreiben die genannten Anwendungsfälle und erläutern, warum sie als Anwendungsfälle ausgesucht wurden.
Bei dem Vergleich zwischen Suchfunktionen werden die Anwendungsfälle herangezogen, um Sucheingaben und erwartete Dokumente zu generieren.
Der genaue Aufbau der Studie wird in \myRef{Kapitel}{chap:aufbau-der-studie} erklärt.

\section{Onboarding im Projekt}
Wenn ein neuer Softwareentwickler in einem Softwareprojekt startet, dann muss er sich zunächst einmal mit dem Projekt vertraut machen.
Das bedeutet, dass er verstehen muss, was das Projekt eigentlich ist.
Er muss verstehen, was das eigentliche Problem des Kunden ist.
Außerdem muss er verstehen wie die Software dieses Problem löst.\\

Für das Onboarding im Projekt muss der Softwareentwickler sehr allgemeine Informationen über das Projekt finden können.
Er könnte Dinge suchen, wie einen Projektüberblick oder ein Glossar.
Neben diesen allgemeinen Informationen muss sich der neue Softwareentwickler mit dem Code vertraut machen.
Er muss verstehen, welche Technologien verwendet werden, welche Best-Practices, Code-Styles, Guidelines, Prozesse und Quality Gates eingehalten werden sollten.
Und er muss verstehen, wie die Software lokal oder in einer Testumgebung ausgeführt werden kann.\\

Die Leitung eines Projektes wünscht sich eine möglichst schnelle Einarbeitung von neuen Mitarbeitern.
Dazu können bereits etablierte Mitarbeiter herangezogen werden, welche den neuen Mitarbeiter bei der Einarbeitung unterstützen.
Der Nachteil besteht darin, dass hierdurch Kapazitäten gebunden werden, welche für die aktive Entwicklung der Software benötigt werden.
Daher kann das Zurückgreifen auf eine Wissensdatenbank durch den neuen Mitarbeiter sinnvoll sein.
Dazu kann der neue Mitarbeiter das Inhaltsverzeichnis der Wissensdatenbank verwenden, wenn es vorhanden ist.
Dieses hilft dem Mitarbeiter dabei Informationen zu bestimmten Themenbereichen zu finden.
Es grenzt die Antwort auf eine Frage auf einen bestimmten Bereich ein, so wie es auch eine Suchfunktion tut.
Nichtsdestotrotz bietet eine Suchfunktion die Möglichkeit dieses Inhaltsverzeichnis automatisch auf Basis einer Sucheingabe zu durchlaufen.
Damit findet der neue Mitarbeiter schneller die Informationen, die er sucht.
Darüber hinaus kann die Suchfunktion spezifischere Ergebnisse liefern.
So kann die Suchfunktion dem Nutzer bereits die relevantesten Stellen in den relevantesten Dokumenten liefern.
Und die Suchfunktion kann dem Nutzer die Informationen aus diesen relevantesten Stellen zusammenfassen.
Voraussetzung hierbei ist eine Suchfunktion, welche gut darin ist die relevantesten Dokumente und die relevantesten Stellen aus diesen Dokumenten zu extrahieren.
Im weiteren Verlauf wird erörtert, welche Ansätze es gibt, um diese Voraussetzung zu erfüllen, und wie aus den Dokumenten die relevantesten Stellen mithilfe von Passage Retrieval ermittelt werden können.
Außerdem wird erläutert, wie Retrieval Augmented Generation die gefundenen Informationen aufbereitet.

\section{Implementierung nach Spezifikation}
\label{chap:implementierung-nach-spezifikation}
Bei der Implementierung von neuen Anforderungen ist es wichtig, dass sich der Softwareentwickler an die Spezifikation hält.
Nur so bekommt der Kunde die Software, die er sich gewünscht hat.
Dazu sollte der Softwareentwickler alle relevanten Dokumente finden, die zu der Spezifikation dazugehören.
Zuerst sollte er die Feature-Spezifikation selbst finden können.
Er sollte die Dokumentation der damit einhergehendenden Prozesse finden, und auch die Domänenobjekte, welche für die Implementierung relevant sind.
Er sollte Diagramme finden können, welche zu dem Anwendungsfall gehören, und auch die weiteren Dokumente, welche den Kontext der Anforderung erläutern.\\

Auch nachdem ein Feature nach Spezifikation umgesetzt wurde, ist es weiterhin wichtig die Spezifikation einfach leicht zu können.
Der Abgleich mit einer Spezifikation ist notwendig, um Testfälle zu schreiben, und zu prüfen, ob das Verhalten der Anwendung korrekt ist.
Es gibt Fehler, welche erkennbar sind, ohne dafür die Spezifikation heranzuziehen.
Eine Anwendung, welche einfriert oder abstürzt ist ein Beispiel für einen solchen Fehler.
Dennoch können Fehler auch domänen- und kontextspezifisch sein, wodurch die Spezifikation als Referenz notwendig ist.
Die Spezifikation zum Abgleich mit dem Verhalten der Software heranzuziehen ist besonders in Randfällen hilfreich.
Angenommen ein Online-Shop bietet einen kostenlosen Versand ab einem Mindestbestellwert an.
Sobald der Preis den Wert von 25\$ übersteigt, ist der Versand gratis.
Der Versand kostet im Normalfall 5\$.
Nun muss definiert werden, ob der Versand in dem Mindestbestelltwert miteinbezogen wird oder nicht.
Wird er miteinbezogen, dann sorgt das dafür, dass ein Produkt, welches 20\$ kostet einen kostenlosen Versand hat, weil der Mindestbestellwert zuzüglich des Versandpreis 25\$ beträgt.
Ob dieses Verhalten gewünscht ist oder nicht, muss in der Spezifikation festgehalten werden.
Wenn nun ein Bugticket dieser Art bei einem Softwareentwickler landet, dann muss er, wie auch bei der Implementierung, alle relevanten Dokumente für die weitere Entwicklung heranziehen.\\

\section{Informationen über das Projektmanagement finden}
Softwareentwickler sollten darüber Bescheid wissen, was in dem Projekt, in dem sie arbeiten, vor sich geht.
Sie sollten über organisatorische Dinge Bescheid wissen, wie die Teamaufteilung und die Zuständigkeiten in den einzelnen Teams und in dem gesamten Projekt.
Das ist wichtig für eine gute Kommunikation und entsprechenden Wissensaustausch zwischen Kollegen.
Außerdem sollten Softwareentwickler darüber Bescheid wissen, wann das nächste Release der Software ansteht.
Diesen Termin sollten sie bei der Planung ihrer Arbeit berücksichtigen, damit sie auch rechtzeitig alle relevanten Features implementiert haben und alle kritischen Fehler behoben haben.
Neben der Einhaltung von Terminen ist die Planung wichtig für die Motivation der Softwareentwickler.
Softwareentwickler sollten die Vision der Software verstehen können und auch die übergreifende Vision des Unternehmens.
Sie sollten verstehen, warum die Arbeit, welche sie erledigen, so wichtig ist.
Und sie sollten verstehen, für wen sie diese Arbeit tun.
Auch diese Faktoren sind wichtig für eine hohe Motivation bei der Arbeit.
Daher ist es so wichtig diese Informationen einfach zugänglich zu machen, also einfach auffindbar zu machen.

In \myRef{Kapitel}{chap:evaluationsmethoden} wird aufgezeigt, wie die Anforderungen der Anwendungsfälle quantifizierbar gemacht werden können.
Das wird dabei helfen nachzuvollziehen, inwieweit die Anforderungen der Anwendungsfälle erreicht wurden, welche genannt wurden.
