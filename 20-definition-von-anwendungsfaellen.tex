\chapter{Definition von Anwendungsfällen}
In diesem Kapitel sollen Anwendungsfälle ausgewählt werden, für welche später Lösungsansätze entwickelt werden.
Die Anwendungsfälle beschreiben die Situationen, in denen Softwareentwickler die Suchfunktionen von Wissensdatenbanken verwenden könnten.
Zur Identifikation von Anwendungsfällen wurde zunächst Literatur herangezogen.
Die Literatur ist bereits unter den verwandten Arbeiten aufgeführt.
In den Verwandten Arbeiten wurden Arbeiten genannt, welche ähnliche Probleme lösen sollen.
Diese Fokussieren sich vorallem auf die "Feature Location", "Bug Localization" und die Traceability zwischen Code und anderen Artefakten.
Mit anderen Worten: Die Suchfunktion soll eine Brücke zwischen Code und Dokumentation herstellen.\\

Jedes der Folgenden Kapitel wird zunächst den jeweiligen Anwendungsfälle näher erläutern.
Nun muss festgelegt werden, wann ein Anwendungsfall als erfüllt betrachtet wird.
Dazu müssen Metriken verwendet werden, um den Erfüllungsgrad der Anwendungsfälle zu quantifizieren.
Hierzu werden Sucheingaben aufgelistet, welche im Kontext des Anwendungsfalls vorkommen könnten.
Das nachfolgende Kapitel "Evaluationsmethoden und -Kriterien" wird die quantifizierung dieser Anforderungen erklären. 

\section{Feature Location}
Bei der Entwicklung eines neuen Features greift der Softwareentwickler also auf die entsprechende Spezifikation zurück.
Dazu muss ihm bekannt sein, wo die Spezifikation zu finden ist.
Nun muss er bei der Entwicklung darauf achten, dass er Best-Practices und Konventionen einhält, sowie die Qualitätsanforderungen.
Eine Qualitätsanforderungen könnte dabei eine vereinbarte Testabdeckung der Software sein.
Der Softwareentwickler muss also bei der Entwicklung eines neuen Features auch diese Informationen einfach finden können.
Und ihm muss klar sein, an welcher Stelle im Code er den neuen Code einbauen sollte.
Das ist Feature Location.

TODO Beispielhafte Sucheingaben auflisten und dafür argumentieren

\section{Bug Localization}
Wenn der Softwareentwickler gerade kein neues Feature implementiert, dann korrigiert er gerade möglicherweise einen Fehler in der Software.
Um einen Fehler überhaupt zu identifizieren, muss aber zuerst wieder die Spezifikation herangezogen werden.
Denn in der Spezifikation wird, wie bereits erwähnt, die gewünschte Funktionsweise der Software beschrieben.
Damit wird auch definiert, was ein fehlerhaftes Verhalten ist, und was ein korrektes Verhalten ist.
Wenn der Softwareentwickler nun ein Fehlerticket erhält, dann muss er die entsprechende Spezifikation zu diesem Fehlerticket finden können.
Und idealerweise wird ihm durch die Suche sogar gleich die betroffene Stelle im Code angezeigt.
Das ist Bug Localization.\\

TODO Beispielhafte Sucheingaben auflisten und dafür argumentieren

\section{Informationen über das Projekt finden}
TODO

\section{Implementierung nach Spezifikation}
TODO

\section{Abgleich mit Spezifikation}
TODO

\section{Onboarding im Projekt}
TODO Best-Practices, Code-Styles, Guidelines, Prozesse, Quality Gates, Deployment, Glossar, Projektmanagement, Testing etc.

\section{Deployment}
TODO

\section{Dokumentation zum besseren Verständnis heranziehen}
TODO Glossar, API-Dokumentation etc.

Im nachfolgenden Kapitel "Evaluationsmethoden und -Kriterien" wird aufgezeigt, wie die Anforderungen der Anwendungsfälle quantifizierbar gemacht werden können.
Das wird dabei helfen nachzuvollziehen, inwieweit die Anforderungen der Anwendungsfälle erreicht wurden, welche genannt wurden.
