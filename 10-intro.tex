%
%
%
\chapter{Einleitung}
\label{chap:intro}

%
%
%
\section{Motivation}
Suchfunktionen sind in vielen Anwendungen vorhanden und können einen großen Einfluss auf die Erfahrung von Nutzern haben. Google ist ein Anbieter einer bekannten Suchfunktion. Sie wird verwendet, um im World Wide Web die passende Website für eine Suchanfrage zu finden. Die Suchanfrage besteht dabei meistens aus lediglich ein bis drei Keywords. Es ist nicht einfach für diese Keywords die Websites zu identifizieren, welche für einen Nutzer am passendsten ist. Auch andere Arten von Anwendungen verwenden Suchfunktionen, darunter Wissensdatenbanken, wie beispielsweise Confluence. In Wissensdatenbanken werden Informationen auf Seiten gespeichert, welche über die Suchfunktion auffindbar sein sollen.
Softwareentwickler haben bei der Arbeit mit Wissensdatenbanken, insbesondere bei der Arbeit mit Spezifikationen und Dokumentationen, oft das Problem, dass die gesuchten Inhalte nicht gefunden werden können. Das kann daran liegen, dass die gefundenen Ergebnisse zu spezifisch sind, z.B. wenn eine allgemeine Definition von Domänenobjekten gesucht wird, aber eine Spezifikation eines Use-Cases gefunden wird, in welchem das Domänenobjekt lediglich erwähnt wird. Außerdem haben manche Wissensdatenbanken Schwierigkeiten, die Semantik einer Suchanfrage zu verstehen. Wird also nach einem Suchbegriff, wie „deployment“ gesucht, dann werden keine Ergebnisse gefunden, welche Wörter, wie „rollout“ beinhalten, auch wenn es sich hierbei um den gleichen Prozess handelt. Softwareentwickler müssen also das genaue Wording in der Dokumentation kennen, um ein sinnvolles Ergebnis angezeigt zu bekommen, und es reicht nicht aus, ein Synonym oder verwandte Wörter oder Konzepte in die Suche einzugeben. Weiterhin werden oftmals Inhalte angezeigt, welche bereits als deprecated markiert sind, und für aktuelle Versionen der Software keine Relevanz mehr haben. 
Ziel dieser Arbeit soll es sein, mithilfe von Wissen über Suchalgorithmen und Algorithmen aus dem Natural Language Processing, diese Probleme zu lösen, sodass Softwareentwickler in Zukunft besser die Inhalte finden, nach denen sie tatsächlich gesucht haben.

%
%
%
\section{Ziele der Arbeit}
TODO

Zur Erreichung des Hauptziels der Arbeit werden folgende Teilziele definiert:

\begin{itemize}
   \item \textbf{Teilziel 1:}
   TODO  
   \item \textbf{Teilziel 2:}
   TODO  
   \item \textbf{Teilziel 3:}
   TODO  
   \item \textbf{Teilziel 4:}
   TODO  
\end{itemize}


%
%
%
\section{Vorgehensweise}
In der Problemstellung wurde behauptet, dass oftmals nicht die Inhalte gefunden werden, welche ein Softwareentwickler bei der Eingabe einer Suche erwarten würde. Diese Behauptung muss zunächst belegt werden. Dazu muss klar sein, welche Informationen ein Softwareentwickler sucht, wenn er die Suchfunktion einer Wissensdatenbank verwendet. Diese sollen in Form von Use-Cases beschrieben werden. Außerdem werden Methoden und Metriken zur Evaluation von Suchfunktionen erläutert. Und es werden verschiedene Implementierungen von Suchfunktionen aufgezählt und kurz beschrieben. Diese verschiedenen Suchfunktionen werden nun für die ausgewählten Use-Cases anhand der zuvor beschriebenen Metriken evaluiert. Um den Rahmen dieser Arbeit nicht zu sprengen, sollen lediglich zwei verschiedene Suchfunktionen miteinander verglichen werden. Zuletzt wird die Implementierung der Suchfunktion erläutert, welche besser abgeschnitten hat.

\subsection*{Adressierung von Teilziel 1}
TODO

\subsection*{Adressierung von Teilziel 2}
TODO

\subsection*{Adressierung von Teilziel 3}
TODO

\subsection*{Adressierung von Teilziel 4}
TODO

\section{Verwandte Arbeiten}
•	ElasticSearch: https://www.elastic.co/de/
•	Apache Lucene: https://lucene.apache.org/

TODO


\section{Abgrenzungen}
Diese Arbeit behandelt keine Large Language Models, wie GPT.