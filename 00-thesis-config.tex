% suppresses page numbering in the header for one page (usage e.g.: \suppressPagenumber{\part{Stand der Forschung}})
\newcommand{\suppressPagenumber}[1]{{\ohead[]{} #1 \ohead[\rightline{\thepage}]{\rightline{\thepage}}}}


% own quotation style for a big single quotation within a block of text
\newcommand{\myQuotation}[2]{\begin{quote}\small{\textit{\enquote{#1}}} -- #2\end{quote}}



% side notes with marginpar
\newcommand{\myNote}[1]{\marginpar{\raggedleft \footnotesize \color{gray} \textit{#1}}}



% cite reference with full title ahead
\newcommand{\myCiteTitle}[1]{\citetitle{#1} (\citeauthor{#1}~\cite{#1})}




% Image/Figure
% Usage: \myImg{caption}{label}{width-factor}{path}
%   Example: \myImg{Beispiel eines Datenflussdiagramms}{fig:dfd}{1}{data-flow-diagram.pdf}
% http://www2.informatik.hu-berlin.de/~piefel/LaTeX-PS/Archive-2004/V08-grafikein.pdf
\newcommand{\myImg}[4]{
    \begin{figure}[h!]
        \centering
        \includegraphics[width=\textwidth * \real{#3}]{./img/#4}
        \caption{\label{#2}#1}
    \end{figure}    
}
\newcommand{\myImgCfg}[5]{
    \begin{figure}[#5]
        \centering
        \includegraphics[width=\textwidth * \real{#3}]{./img/#4}
        \caption{\label{#2}#1}
    \end{figure}    
}

% Shortcut for "..Abbildung 2 zeigt.." via "..\myImgRef{fig:screenshot} zeigt.."
\newcommand{\myImgRef}[1]{\hyperref[#1]{Abbildung~\ref*{#1}}}



% Shortcut for "..Abschnitt 2 beschreibt.." via "..\myRef{Abschnitt}{chap:myrefchapter}.."
\newcommand{\myRef}[2]{\hyperref[#2]{#1~\ref*{#2}}}


% Shortcut for "..auf Seite 7.." via "..\myPageRef{fig:myfigure}.."
\newcommand{\myPageRef}[1]{\hyperref[#1]{Seite~\pageref*{#1}}}



% Shortcuts for frequently used strings
\newcommand{\sczb}{z.\,B. } % zum Beispiel; added space because those phrases will never be used at the end of a sentence
\newcommand{\scdh}{d.\,h. } % das heißt
\newcommand{\scua}{u.\,a. } % unter anderem






% Mathematics
\newcommand{\powerset}[1]{\mathcal P~\left({#1}\right)}
%\newcommand{\powerset}[1]{\wp \left({#1}\right)}



% Font
% changes font family ptm = Times, phv = Helvetica,.. http://de.wikibooks.org/wiki/LaTeX-W%C3%B6rterbuch:_fontfamily
%\renewcommand{\familydefault}{phv}



% colored box
% see LaTeX - fcolorbox.png
% http://www.lessjunkmorefunk.de/de/node/10
%\definecolor{Gray}{gray}{0.9}
%\newcommand{\inhalt}[1]{\fcolorbox{black}{Gray}{\parbox{\textwidth}{#1}}}